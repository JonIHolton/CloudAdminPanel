\documentclass[11pt]{article}
\usepackage{geometry}
\usepackage{graphicx}
\usepackage{hyperref}
\usepackage{tabularx}
\usepackage{authblk}

\geometry{a4paper, margin=1in}

\title{ITSA - PROJECT NAME}
\author{
  Author One \\
  Department of Computer Science \\
  University A \\
  \texttt{author1@email.com}
  \and
  Author Two \\
  Department of Engineering \\
  University B \\
  \texttt{author2@email.com}
  \and
  Author Three \\
  Department of Physics \\
  University C \\
  \texttt{author3@email.com}
}
\date{\today}

\begin{document}

\maketitle
\newpage
\tableofcontents
\newpage

\section{Introduction}
\subsection{Purpose}
This document serves as a comprehensive report for the Vulnerability Assessment and Penetration Testing (VAPT), Design, and Software Development Life Cycle (SDLC) of the project.

\subsection{Scope}
The scope of this report encompasses the evaluation of the project's security through VAPT, design considerations, and an overview of the SDLC.

\subsection{Definitions, Acronyms, and Abbreviations}
\begin{tabularx}{\textwidth}{|l|X|}
    \hline
    \textbf{Term} & \textbf{Definition} \\
    \hline
    VAPT & Vulnerability Assessment and Penetration Testing \\
    \hline
    SDLC & Software Development Life Cycle \\
    \hline
    API & Application Programming Interface \\
    \hline
\end{tabularx}

\section{Vulnerability Assessment and Penetration Testing (VAPT)}
\subsection{Overview}
The VAPT process involves evaluating the project's security measures to identify vulnerabilities and potential threats.

\subsection{VAPT Methodology}
Our VAPT methodology follows industry best practices, including scanning, manual testing, and reporting.

\subsection{Tools and Technologies Used}
We utilized various tools and technologies for VAPT, including Nessus, Burp Suite, and Wireshark.

\subsection{Vulnerability Assessment Report}
\subsubsection{Findings}
During the assessment, multiple vulnerabilities were identified and categorized.

\subsubsection{Risk Analysis}
The identified vulnerabilities were analyzed for their potential impact and likelihood.

\subsection{Penetration Testing Report}
\subsubsection{Test Scenarios}
Several penetration testing scenarios were executed to assess the project's security posture.

\subsubsection{Test Results}
The results of the penetration testing revealed potential weaknesses and areas of improvement.

\section{Use Cases}
\subsection{Key Use Cases}
\begin{tabularx}{\textwidth}{|l|X|}
    \hline
    \textbf{Use Case Title} & Do Magic() \\
    \hline
    \textbf{Use Case ID} & 1 \\
    \hline
    \textbf{Description} & Lorem Ipsum.... \\
    \hline 
    \textbf{Alternate flow:} & Lorem Impsum.... \\
    \hline 
\end{tabularx}

\section{Conclusion}
In conclusion, this report highlights the results of the VAPT, key use cases, and provides an overview of the project's security and design considerations.

\section{References}
- [Reference 1]
- [Reference 2]

\end{document}
